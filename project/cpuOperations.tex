\subsection{Measurement Overhead}

In order to accurately measure execution time we implemented a timing
methodology that reads the hardware time-stamp counter before and after a
procedure (by procedure here we mean lines of C or assembly code we wish to
profile, not necessarily a procedure call).  The time-stemp is read immediately
before the procedure is called and again after the procedure exits. The two
output values correspond to the start and end times of the procedure being
executed. The initial time is subtracted from the final time to calculate the
number of cycles that have elapsed.

The implementation for reading the time-stamp makes use of the assembly
instructions CPUID, RDTSC, and RDTSCP. CPUID and RDTSCP are used to force
serialization of instructions so that instructions which are not part of the
code being instrumented do not pollute our measurements due to out-of-order
execution.\cite{intel}

We run the timing harness and procedure being timed in a loop of 10,000
iterations. The execution time is collected on each iteration of the loop. We
use the median execution time over all iterations as the final result in order
to reduce the influence that extreme values caused by context switches,
interrupts and other exceptional events have on our experiment. 

\subsubsection{Measuring time overhead} 

We estimate that the hardware overhead of reading the time-stamp registers and
calculating the elapsed time is 2 cycles. Since the time-stamp is implemented in
registers it would require 1 cycle to read the current time. 

We implemented the reading of the start and end time with inline assembly. The
code to read the start time and the code to read the end time each consists of
four assembly instructions. This includes the calls to CPUID which act as a
fence to force sequential execution of code.  Since the time-stamp register is
64-bits we retrieve the upper and lower 32-bits into two separate registers. We
reform the original 64-bit value by using a shift and bitwise or
operation.Finally,  The two 64-bit values are subtracted to produce the final
elapsed time in cycles. We estimate the software overhead to be 15 cycles.

The total overhead of our measuring mechanism is estimated to be 17 cycles.

In order to gain the most accurate measure of the timing overhead we disabled
additional cores and frequency scaling within the bios.

For computing the overhead of reading time, we ran the timing harness that was
previously described without any intervening code.  Our results show that the
overhead of reading time to be 40 cycles. 

Our results show that we underestimated the overhead of our timing mechanism. We
believe this is due to our assumption that the instructions CPUID and RDTSCP,
which is a serializing version of RDTSC, executes in a single cycle. We were
able to verify this by removing the CPUID calls and replacing RDTSCP with RDTSC.
The experiment without the serializing instructions showed that the overhead of
the timing mechanism was 22 cycles which is closer to our estimated result.

\subsubsection{Measuring loop overhead}

For the looping overhead,  we have to be careful that our looping procedure does
not get optimized out either by the compiler or the assembly interpreter. For
example, an empty loop would be quickly optimized with a constant propagation
compiler pass. As such, we inserted a call to the \texttt{rand()} C function
(for pseudo-random numbers) which guarantees the loop actually has to execute
our specified number of times in order to obtain a correct result. In order to
ensure that no optimizations are performed,  we disable compiler optimizations
by using the "-o0" flag and implemented our procedure under test using inline
assembly.

The procedure under test is a loop, written in 5 lines of assembly code, that
runs for 10 iterations.  We estimate that a loop would not add any hardware
overhead. In terms of software overhead, we estimate that 5 additional cycles
will be required for each iteration of the loop. This number was arrived at by
examining the assembly code that implements the loop and assuming that each
instruction executes in a single cycle.

Our results show that our estimate of the loop overhead was very close to the
measured loop overhead of 4 cycles. The discrepency is the result of counting an
instruction that initializes the number of loops to perform by moving a
parameter into a register. This instruction does not participate in the actual
loop.

An interesting result that we discovered is that after several iterations of the
timing procedure (i.e. timing and loop of ten iterations) the loop overhead is
reduced to a single cycle. We suspect this is due to optimizations which are
occuring in the hardware in spite of disabling optimizations during compile
time.

\subsection{Procedure call overhead}

Here as well, we had to be careful about compiler optimizations. Again we used
the \texttt{rand()} function. The procedures we profile makes a unique call to
the rand() function and returns (and never uses any of its arguments). We create
eight such procedures (with 0 to 7 arguments).

Using our methodology, we measured the average time required for a single call
to the \texttt{rand()} function and each of our eight procedures (and compute
the difference). Measurements available soon.

\subsection{System call overhead}

\subsection{Task Creation Time}

In order to compare the execution time of a kernel thread and a process, we
setup a (slow) fibonacci function which would be the only thing the kernel
thread or process would execute.

We were able to profile kernel threads using our methodology. The procedure
between the two hardware counters simply creates a kernel thread which runs the
fibonacci function waits for that thread to complete. The average time spent by
the thread was 32264 cycles.

While kernel threads can be measured using our methodology, processes cannot be
created with a kernel module, since there is no fork or exec call in the linux
kernel API. As such for this task, we had to go out of kernel mode and write a
user program at the possible cost of loss of precision (however, this gave us
the opportunity to shut down compiler optimizations!). This time, we use a linux
\texttt{clone()} system call and had the child process execute the fibonacci
function. Measurements available soon.

\subsection{Context Switch Time} \subsubsection{Process Context Switch}
\subsubsection{Thread Context Switch}
